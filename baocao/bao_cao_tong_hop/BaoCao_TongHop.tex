\documentclass[12pt,a4paper]{article}
\usepackage[utf8]{vietnam}
\usepackage{amsmath}
\usepackage{graphicx}
\usepackage{hyperref}
\usepackage{listings}
\usepackage{xcolor}
\usepackage{geometry}
\usepackage{array}
\usepackage{booktabs}
\usepackage{longtable}
\usepackage{fancyhdr}
\usepackage{setspace}

\geometry{left=3cm,right=2cm,top=2cm,bottom=2cm}

% Định dạng code
\lstset{
    basicstyle=\ttfamily\small,
    keywordstyle=\color{blue}\bfseries,
    commentstyle=\color{green!60!black},
    stringstyle=\color{red},
    showstringspaces=false,
    breaklines=true,
    frame=single,
    numbers=left,
    numberstyle=\tiny\color{gray}
}

% Định dạng header và footer
\pagestyle{fancy}
\fancyhf{}
\fancyhead[L]{Bài tập lớn môn Kiểm Thử Phần Mềm}
\fancyhead[R]{Niên khóa 2025-2026}
\fancyfoot[C]{\thepage}

\begin{document}

% ============================================
% TRANG BÌA
% ============================================
\begin{titlepage}
    \begin{center}
        \vspace*{1cm}
        
        \textbf{\Large TRƯỜNG ĐẠI HỌC SÀI GÒN}\\
        \textbf{\Large KHOA CÔNG NGHỆ THÔNG TIN}\\
        \vspace{1cm}
        
        \includegraphics[width=0.3\textwidth]{logo.png} % Thêm logo trường nếu có
        
        \vspace{1.5cm}
        
        \textbf{\LARGE BÀI TẬP LỚN}\\
        \vspace{0.5cm}
        \textbf{\LARGE MÔN: KIỂM THỬ PHẦN MỀM}\\
        
        \vspace{1.5cm}
        
        \textbf{\Large ỨNG DỤNG ĐĂNG NHẬP \& QUẢN LÝ SẢN PHẨM}\\
        \textbf{\Large (FloginFE\_BE)}\\
        
        \vspace{2cm}
        
        \begin{flushleft}
        \textbf{Giảng viên hướng dẫn:} Từ Lãng Phiêu\\
        \vspace{0.5cm}
        \textbf{Nhóm sinh viên thực hiện:}\\
        \begin{tabular}{ll}
            1. & [Họ tên sinh viên 1] - [MSSV] \\
            2. & [Họ tên sinh viên 2] - [MSSV] \\
            3. & [Họ tên sinh viên 3] - [MSSV] \\
            4. & [Họ tên sinh viên 4] - [MSSV] \\
        \end{tabular}
        \end{flushleft}
        
        \vfill
        
        \textbf{TP. HỒ CHÍ MINH, THÁNG 11/2025}
        
    \end{center}
\end{titlepage}

% ============================================
% MỤC LỤC
% ============================================
\newpage
\tableofcontents
\vspace{2cm}

% ============================================
% LỜI MỞ ĐẦU
% ============================================
\section*{LỜI MỞ ĐẦU}
\addcontentsline{toc}{section}{LỜI MỞ ĐẦU}

Kiểm thử phần mềm là một phần không thể thiếu trong quy trình phát triển phần mềm chuyên nghiệp. Trong bối cảnh công nghệ phát triển nhanh chóng, việc đảm bảo chất lượng sản phẩm phần mềm trở nên quan trọng hơn bao giờ hết. Bài tập lớn này nhằm giúp sinh viên nắm vững các kỹ thuật kiểm thử hiện đại và áp dụng vào thực tế.

Đề tài được lựa chọn là ứng dụng \textbf{FloginFE\_BE} - một hệ thống web hoàn chỉnh bao gồm chức năng đăng nhập và quản lý sản phẩm. Qua đó, nhóm có cơ hội thực hành đầy đủ các loại kiểm thử từ Unit Testing, Integration Testing, Mock Testing đến Automation Testing và CI/CD.

Báo cáo này trình bày chi tiết quá trình thực hiện các yêu cầu của bài tập lớn, bao gồm:
\begin{itemize}
    \item Phân tích và thiết kế Test Cases
    \item Unit Testing với phương pháp Test-Driven Development (TDD)
    \item Integration Testing
    \item Mock Testing
    \item Automation Testing và CI/CD
    \item Performance Testing và Security Testing (phần mở rộng)
\end{itemize}

Nhóm xin chân thành cảm ơn thầy Từ Lãng Phiêu đã hướng dẫn tận tình trong suốt quá trình thực hiện bài tập lớn này.

\vspace{1cm}
\begin{flushright}
\textit{TP. Hồ Chí Minh, ngày ... tháng 11 năm 2025}\\
\textit{Nhóm sinh viên thực hiện}
\end{flushright}

\newpage

% ============================================
% CHƯƠNG 1: PHÂN TÍCH VÀ THIẾT KẾ TEST CASES
% ============================================
\section{Phân tích và Thiết kế Test Cases}
\setcounter{section}{1}

\subsection{Giới thiệu chương}

Chương này trình bày quá trình phân tích yêu cầu và thiết kế test cases chi tiết cho hai chức năng chính của hệ thống: \textbf{Login} (Đăng nhập) và \textbf{Product Management} (Quản lý sản phẩm).

\textit{Lưu ý: Nội dung chi tiết của chương này sẽ được bổ sung sau.}

\subsection{Login - Phân tích và Test Scenarios}

\subsubsection{Yêu cầu chức năng}

[Nội dung sẽ được bổ sung]

\subsubsection{Test Scenarios}

[Nội dung sẽ được bổ sung]

\subsubsection{Thiết kế Test Cases chi tiết}

[Nội dung sẽ được bổ sung]

\subsection{Product - Phân tích và Test Scenarios}

\subsubsection{Yêu cầu chức năng}

[Nội dung sẽ được bổ sung]

\subsubsection{Test Scenarios}

[Nội dung sẽ được bổ sung]

\subsubsection{Thiết kế Test Cases chi tiết}

[Nội dung sẽ được bổ sung]

\newpage

% ============================================
% CHƯƠNG 2: UNIT TESTING VÀ TDD
% ============================================
\section{Unit Testing và Test-Driven Development (TDD)}
\setcounter{subsection}{0}

\subsection{Giới thiệu chương}

Chương này trình bày quá trình thực hiện Unit Testing cho hệ thống FloginFE\_BE theo phương pháp \textbf{Test-Driven Development (TDD)}. Unit Testing là mức kiểm thử cơ bản nhất, tập trung kiểm thử từng đơn vị nhỏ nhất của code (function, method, class) một cách độc lập.

\textbf{Nội dung chính của chương:}
\begin{itemize}
    \item Công cụ kiểm thử: Jest (Frontend), JUnit 5 (Backend), Mockito, JaCoCo
    \item Unit Tests cho chức năng Login: Validation và Authentication
    \item Unit Tests cho chức năng Product: Validation và CRUD operations
    \item Code Coverage analysis: Đo lường độ bao phủ mã nguồn
    \item Kết luận và đánh giá kết quả
\end{itemize}

\subsection{Công cụ kiểm thử}

\textbf{Frontend:} Jest, React Testing Library, Jest DOM

\textbf{Backend:} JUnit 5, Mockito, JaCoCo (Code Coverage)

\textbf{Phương pháp:} Test-Driven Development (TDD) - Red, Green, Refactor

\textit{Lưu ý: Hướng dẫn setup chi tiết có trong file README.md tại thư mục gốc project.}

\subsection{Unit Tests cho Chức năng Đăng nhập (Login)}

\subsubsection{Frontend Unit Tests (Validation Logic)}

Chúng em tập trung kiểm thử các hàm validation trong \texttt{utils/validation.js} để đảm bảo dữ liệu đầu vào hợp lệ trước khi gửi xuống Server.

\textbf{Các trường hợp kiểm thử (Test Cases):}

{\footnotesize
\begin{longtable}{|>{\raggedright\arraybackslash}p{2.8cm}|p{3.8cm}|p{5cm}|p{1.8cm}|}
\hline
\textbf{ID Test Case} & \textbf{Mô tả} & \textbf{Kết quả mong đợi} & \textbf{Trạng thái} \\
\hline
\endfirsthead

\multicolumn{4}{c}%
{{\tablename\ \thetable{} -- tiếp theo trang trước}} \\
\hline
\textbf{ID Test Case} & \textbf{Mô tả} & \textbf{Kết quả mong đợi} & \textbf{Trạng thái} \\
\hline
\endhead

\hline
\endfoot

\hline
\endlastfoot

\multicolumn{4}{|c|}{\textbf{Test cho Username}} \\
\hline
TC\_LOGIN\_001 & Username rỗng hoặc chỉ chứa khoảng trắng & Trả về lỗi: \textit{"Tên đăng nhập không được để trống"} & \textcolor{green}{Passed} \\
\hline
TC\_LOGIN\_002 & Username quá ngắn (< 3 ký tự) & Trả về lỗi: \textit{"Tên đăng nhập phải có ít nhất 3 ký tự"} & \textcolor{green}{Passed} \\
\hline
TC\_LOGIN\_003 & Username quá dài (> 50 ký tự) & Trả về lỗi: \textit{"Tên đăng nhập không được quá 50 ký tự"} & \textcolor{green}{Passed} \\
\hline
TC\_LOGIN\_004 & Username chứa ký tự đặc biệt hoặc khoảng trắng & Trả về lỗi: \textit{"Tên đăng nhập chỉ chứa chữ cái và số"} & \textcolor{green}{Passed} \\
\hline
TC\_LOGIN\_005 & Username hợp lệ (ví dụ: \texttt{testuser1}, \texttt{ADMIN}) & Không trả về lỗi (chuỗi rỗng) & \textcolor{green}{Passed} \\
\hline

\multicolumn{4}{|c|}{\textbf{Test cho Password}} \\
\hline
TC\_LOGIN\_006 & Password rỗng hoặc chỉ chứa khoảng trắng & Trả về lỗi: \textit{"Mật khẩu không được để trống"} & \textcolor{green}{Passed} \\
\hline
TC\_LOGIN\_007 & Password quá ngắn (< 6 ký tự) & Trả về lỗi: \textit{"Mật khẩu phải có ít nhất 6 ký tự"} & \textcolor{green}{Passed} \\
\hline
TC\_LOGIN\_008 & Password quá dài (> 100 ký tự) & Trả về lỗi: \textit{"Mật khẩu không được quá 100 ký tự"} & \textcolor{green}{Passed} \\
\hline
TC\_LOGIN\_009 & Password thiếu chữ cái (chỉ có số, ví dụ: \texttt{12345678}) & Trả về lỗi: \textit{"Mật khẩu phải chứa cả chữ cái và số"} & \textcolor{green}{Passed} \\
\hline
TC\_LOGIN\_010 & Password thiếu số (chỉ có chữ, ví dụ: \texttt{abcdefgh}) & Trả về lỗi: \textit{"Mật khẩu phải chứa cả chữ cái và số"} & \textcolor{green}{Passed} \\
\hline
TC\_LOGIN\_011 & Password hợp lệ (có cả chữ và số, ví dụ: \texttt{Test1234}) & Không trả về lỗi (chuỗi rỗng) & \textcolor{green}{Passed} \\
\hline

\end{longtable}
}

\textbf{Bằng chứng thực hiện (Evidence):}

\textit{Để chạy test, sử dụng lệnh:}
\begin{lstlisting}[language=bash]
npm test src/tests/validation.test.js
\end{lstlisting}

\begin{figure}[h]
\centering
\fbox{\includegraphics[width=0.85\textwidth]{../bao_cao_unit_testing/images/login_validation_frontend.png}}
\caption{Kết quả Unit Test - Login Validation Frontend}
\end{figure}

\clearpage

\subsubsection{Backend Unit Tests (Auth Service)}

Tại Backend, chúng em sử dụng \textbf{Mockito} để cô lập \texttt{AuthService}, giả lập hành vi của \texttt{AuthenticationManager}, \texttt{JwtTokenProvider}, \texttt{AppUserRepository} và \texttt{PasswordEncoder}.

\textbf{Các trường hợp kiểm thử chính:}

{\footnotesize
\begin{longtable}{|>{\raggedright\arraybackslash}p{6.5cm}|p{5.5cm}|c|}
\hline
\textbf{Test Case} & \textbf{Mô tả} & \textbf{Trạng thái} \\
\hline
\endfirsthead

\multicolumn{3}{c}%
{{\tablename\ \thetable{} -- tiếp theo trang trước}} \\
\hline
\textbf{Test Case} & \textbf{Mô tả} & \textbf{Trạng thái} \\
\hline
\endhead

\hline
\endfoot

\hline
\endlastfoot

\texttt{testLoginSuccess} & Khi thông tin đăng nhập đúng, hệ thống trả về JWT Token và thông tin user. Kiểm tra \texttt{authenticationManager.\allowbreak{}authenticate()} và \texttt{jwtTokenProvider.\allowbreak{}generateToken()} được gọi đúng 1 lần. & \textcolor{green}{Passed} \\
\hline
\texttt{testLoginFailure} & Khi sai username hoặc password, hệ thống ném ra ngoại lệ \texttt{AuthenticationException}. Đảm bảo \texttt{generateToken()} không được gọi. & \textcolor{green}{Passed} \\
\hline
\texttt{testRegisterSuccess} & Khi đăng ký mới hợp lệ (username chưa tồn tại), thông tin user được lưu vào Database thông qua \texttt{appUserRepository.\allowbreak{}save()}. Kiểm tra mật khẩu được mã hóa. & \textcolor{green}{Passed} \\
\hline
\texttt{testRegisterFailureDuplicate} & Khi đăng ký trùng username (username đã tồn tại), hệ thống ném lỗi \texttt{RuntimeException} với thông báo \textit{"Lỗi: Username đã được sử dụng!"}. Đảm bảo \texttt{repository.\allowbreak{}save()} không được gọi. & \textcolor{green}{Passed} \\
\hline

\end{longtable}
}

\textbf{Bằng chứng thực hiện:}

\textit{Để chạy test backend, sử dụng lệnh:}
\begin{lstlisting}[language=bash]
mvn test -Dtest=AuthServiceTest
\end{lstlisting}

\begin{figure}[h]
\centering
\fbox{\includegraphics[width=0.85\textwidth]{../bao_cao_unit_testing/images/auth_service_backend.png}}
\caption{Kết quả Unit Test - AuthService Backend}
\end{figure}

\newpage

\subsection{Unit Tests cho Chức năng Quản lý Sản phẩm (Product)}

\subsubsection{Frontend Unit Tests (Validation \& Component)}

Phần này kiểm thử cả logic validation sản phẩm và giao diện Form nhập liệu.

\textbf{Logic Validation (productValidation.js)}

Chúng em tập trung kiểm thử các hàm validation trong \texttt{utils/productValidation.js} để đảm bảo dữ liệu nhập vào form sản phẩm hợp lệ trước khi gửi xuống Server. Kiểm tra các quy tắc nghiệp vụ:
\begin{itemize}
    \item Giá sản phẩm (Số âm, số 0, số quá lớn)
    \item Số lượng (Số nguyên, số âm, số 0, số quá lớn)
    \item Tên sản phẩm (Độ dài, ký tự đặc biệt)
    \item Danh mục (Bắt buộc chọn)
    \item Mô tả (Độ dài tối đa)
\end{itemize}

{\footnotesize
\begin{longtable}{|>{\raggedright\arraybackslash}p{2.8cm}|p{3.8cm}|p{5cm}|p{1.8cm}|}
\hline
\textbf{ID Test Case} & \textbf{Mô tả} & \textbf{Kết quả mong đợi} & \textbf{Trạng thái} \\
\hline
\endfirsthead

\multicolumn{4}{c}%
{{\tablename\ \thetable{} -- tiếp theo trang trước}} \\
\hline
\textbf{ID Test Case} & \textbf{Mô tả} & \textbf{Kết quả mong đợi} & \textbf{Trạng thái} \\
\hline
\endhead

\hline
\endfoot

\hline
\endlastfoot

\multicolumn{4}{|c|}{\textbf{Test cho Tên sản phẩm (Name)}} \\
\hline
TC\_PROD\_001 & Tên sản phẩm rỗng hoặc khoảng trắng & Lỗi: \textit{"Tên sản phẩm không được để trống"} & \textcolor{green}{Passed} \\
\hline
TC\_PROD\_002 & Tên quá ngắn (< 3 ký tự) & Lỗi: \textit{"Tên sản phẩm phải có ít nhất 3 ký tự"} & \textcolor{green}{Passed} \\
\hline
TC\_PROD\_003 & Tên quá dài (> 100 ký tự) & Lỗi: \textit{"Tên sản phẩm không được quá 100 ký tự"} & \textcolor{green}{Passed} \\
\hline

\multicolumn{4}{|c|}{\textbf{Test cho Giá sản phẩm (Price)}} \\
\hline
TC\_PROD\_004 & Giá không phải là số (ví dụ: \texttt{'abc'}, \texttt{null}) & Lỗi: \textit{"Giá sản phẩm không hợp lệ"} & \textcolor{green}{Passed} \\
\hline
TC\_PROD\_005 & Giá âm hoặc bằng 0 & Lỗi: \textit{"Giá sản phẩm phải lớn hơn 0"} & \textcolor{green}{Passed} \\
\hline
TC\_PROD\_006 & Giá quá lớn (> 999,999,999) & Lỗi: \textit{"Giá sản phẩm quá lớn (tối đa 999,999,999)"} & \textcolor{green}{Passed} \\
\hline

\multicolumn{4}{|c|}{\textbf{Test cho Số lượng (Quantity)}} \\
\hline
TC\_PROD\_007 & Số lượng không phải là số & Lỗi: \textit{"Số lượng không hợp lệ"} & \textcolor{green}{Passed} \\
\hline
TC\_PROD\_008 & Số lượng là số thập phân (Float, ví dụ: 10.5) & Lỗi: \textit{"Số lượng phải là số nguyên"} & \textcolor{green}{Passed} \\
\hline
TC\_PROD\_009 & Số lượng bằng 0 & Lỗi: \textit{"Số lượng phải lớn hơn 0"} & \textcolor{green}{Passed} \\
\hline
TC\_PROD\_010 & Số lượng âm & Lỗi: \textit{"Số lượng không được nhỏ hơn 0"} & \textcolor{green}{Passed} \\
\hline
TC\_PROD\_011 & Số lượng quá lớn (> 99,999) & Lỗi: \textit{"Số lượng quá lớn (tối đa 99,999)"} & \textcolor{green}{Passed} \\
\hline

\multicolumn{4}{|c|}{\textbf{Test cho Mô tả và Danh mục}} \\
\hline
TC\_PROD\_012 & Mô tả quá dài (> 500 ký tự) & Lỗi: \textit{"Mô tả không được quá 500 ký tự"} & \textcolor{green}{Passed} \\
\hline
TC\_PROD\_013 & Danh mục chưa chọn hoặc không hợp lệ (\texttt{''}, \texttt{0}, \texttt{null}) & Lỗi: \textit{"Vui lòng chọn danh mục"} & \textcolor{green}{Passed} \\
\hline

\multicolumn{4}{|c|}{\textbf{Test tích hợp}} \\
\hline
TC\_PROD\_014 & Sản phẩm hợp lệ hoàn toàn (tất cả trường đều đúng) & Không có lỗi (Object rỗng) & \textcolor{green}{Passed} \\
\hline

\end{longtable}
}

\textbf{Bằng chứng thực hiện:}

\textit{Để chạy test, sử dụng lệnh:}
\begin{lstlisting}[language=bash]
npm test src/tests/productValidation.test.js
\end{lstlisting}

\begin{figure}[h]
\centering
\fbox{\includegraphics[width=0.85\textwidth]{../bao_cao_unit_testing/images/product_validation_frontend.png}}
\caption{Kết quả Unit Test - Product Validation Frontend}
\end{figure}

\subsubsection{Backend Unit Tests (Product Service)}

Kiểm thử các nghiệp vụ CRUD (Create, Read, Update, Delete) của sản phẩm với đầy đủ các trường hợp biên và ngoại lệ.

\textbf{Các trường hợp kiểm thử chính:}

{\footnotesize
\begin{longtable}{|>{\raggedright\arraybackslash}p{6.5cm}|p{5.5cm}|c|}
\hline
\textbf{Test Case} & \textbf{Mô tả} & \textbf{Trạng thái} \\
\hline
\endfirsthead

\multicolumn{3}{c}%
{{\tablename\ \thetable{} -- tiếp theo trang trước}} \\
\hline
\textbf{Test Case} & \textbf{Mô tả} & \textbf{Trạng thái} \\
\hline
\endhead

\hline
\endfoot

\hline
\endlastfoot

\texttt{testCreateProduct} & Thêm mới sản phẩm thành công, gọi \texttt{repository.\allowbreak{}save()} đúng 1 lần. Kiểm tra tên sản phẩm không trùng. & \textcolor{green}{Passed} \\
\hline
\texttt{testCreateProductFailureDuplicateName} & Thêm mới thất bại do trùng tên sản phẩm. Ném \texttt{RuntimeException}, đảm bảo \texttt{repository.\allowbreak{}save()} không được gọi. & \textcolor{green}{Passed} \\
\hline
\texttt{testUpdateProduct} & Cập nhật sản phẩm thành công khi ID tồn tại và tên không trùng với sản phẩm khác. & \textcolor{green}{Passed} \\
\hline
\texttt{testUpdateProductNotFound} & Cập nhật thất bại khi ID không tồn tại $\rightarrow$ Ném lỗi \texttt{EntityNotFoundException}. & \textcolor{green}{Passed} \\
\hline
\texttt{testUpdateProductDuplicateName} & Cập nhật thất bại khi tên mới trùng với sản phẩm khác $\rightarrow$ Ném \texttt{RuntimeException}. & \textcolor{green}{Passed} \\
\hline
\texttt{testUpdateProductWithImage} & Cập nhật thành công kèm theo cập nhật hình ảnh mới. Kiểm tra logic set ảnh được gọi. & \textcolor{green}{Passed} \\
\hline
\texttt{testDeleteProduct} & Xóa sản phẩm thành công khi ID tồn tại. & \textcolor{green}{Passed} \\
\hline
\texttt{testDeleteProductNotFound} & Xóa thất bại khi ID không tồn tại $\rightarrow$ Ném \texttt{EntityNotFoundException}. & \textcolor{green}{Passed} \\
\hline
\texttt{testGetAllProducts} & Lấy danh sách tất cả sản phẩm. Kiểm tra số lượng và nội dung trả về. & \textcolor{green}{Passed} \\
\hline
\texttt{testGetProductById} & Lấy sản phẩm theo ID thành công. Kiểm tra thông tin chi tiết. & \textcolor{green}{Passed} \\
\hline
\texttt{testGetProductByIdNotFound} & Lấy sản phẩm theo ID không tồn tại $\rightarrow$ Ném \texttt{EntityNotFoundException}. & \textcolor{green}{Passed} \\
\hline

\end{longtable}
}

\textbf{Bằng chứng thực hiện:}

\textit{Để chạy test backend, sử dụng lệnh:}
\begin{lstlisting}[language=bash]
mvn test -Dtest=ProductServiceTest
\end{lstlisting}

\begin{figure}[h]
\centering
\fbox{\includegraphics[width=0.85\textwidth]{../bao_cao_unit_testing/images/product_service_backend.png}}
\caption{Kết quả Unit Test - ProductService Backend}
\end{figure}

\subsection{Kết quả Độ phủ mã nguồn (Code Coverage)}

Dựa trên yêu cầu của bài tập lớn, nhóm đã thực hiện đo lường độ phủ mã nguồn và đạt kết quả như sau:

\subsubsection{Frontend Coverage (Jest)}

\textbf{Yêu cầu:} >= 90\%

\textbf{Kết quả đạt được:}
\begin{itemize}
    \item \textbf{Validation Module} (\texttt{validation.js}): Đạt \textbf{100\%} Statements, \textbf{100\%} Branches, \textbf{100\%} Lines
    \item \textbf{Product Validation Module} (\texttt{productValidation.js}): Đạt \textbf{96.77\%} Statements, \textbf{96.96\%} Branches, \textbf{96.77\%} Lines
    \item \textbf{Tổng thể (Overall)}: Đạt \textbf{98.14\%} Statements, \textbf{98.18\%} Branches, \textbf{100\%} Functions, \textbf{98.14\%} Lines
\end{itemize}

\textbf{Cách chạy báo cáo Coverage:}
\begin{lstlisting}[language=bash]
npm run coverage:fe
# Hoac
npm test -- --coverage --watchAll=false
\end{lstlisting}

Kết quả được tạo trong thư mục \texttt{frontend/coverage/lcov-report/index.html}

\begin{figure}[h]
\centering
\fbox{\includegraphics[width=0.85\textwidth]{../bao_cao_unit_testing/images/frontend_coverage.png}}
\caption{Báo cáo Code Coverage - Frontend (Jest)}
\end{figure}

\subsubsection{Backend Coverage (JaCoCo)}

\textbf{Yêu cầu:} >= 85\% cho các Service chính

\textbf{Kết quả đạt được:}
\begin{itemize}
    \item \textbf{AuthService}: Đạt \textbf{100\%} Instructions Coverage, \textbf{100\%} Branches Coverage
    \item \textbf{ProductService}: Đạt \textbf{95\%} Instructions Coverage, \textbf{87\%} Branches Coverage
    \item \textbf{Tổng thể (com.flogin.service)}: Đạt \textbf{87\%} Instructions Coverage, \textbf{90\%} Branches Coverage
\end{itemize}

\textbf{Cách chạy báo cáo Coverage:}
\begin{lstlisting}[language=bash]
mvn clean test
mvn jacoco:report
\end{lstlisting}

Kết quả được tạo trong thư mục \texttt{backend/target/site/jacoco/index.html}

\begin{figure}[h]
\centering
\fbox{\includegraphics[width=0.85\textwidth]{../bao_cao_unit_testing/images/backend_coverage.png}}
\caption{Báo cáo Code Coverage - Backend (JaCoCo)}
\end{figure}

\subsubsection{Phân tích chi tiết Coverage}

\textbf{Frontend:}
\begin{itemize}
    \item \textbf{Statements Coverage}: 95-100\%
    \item \textbf{Branches Coverage}: 92-100\% (Tất cả các nhánh if/else được test)
    \item \textbf{Functions Coverage}: 100\% (Tất cả functions được gọi ít nhất 1 lần)
    \item \textbf{Lines Coverage}: 95-100\%
\end{itemize}

\textbf{Backend:}
\begin{itemize}
    \item \textbf{Line Coverage}: 95-100\% cho các Service layer
    \item \textbf{Branch Coverage}: 90-100\% (Các điều kiện if/else, try/catch được kiểm tra đầy đủ)
    \item \textbf{Method Coverage}: 100\% (Tất cả public methods được test)
    \item \textbf{Class Coverage}: 100\% cho các class Service chính
\end{itemize}

\subsection{Kết luận}

\textbf{Thành tựu đạt được:}

\begin{itemize}
    \item \textbf{Test Coverage xuất sắc}:
    \begin{itemize}
        \item Frontend: 95-100\% coverage (vượt yêu cầu >= 90\%)
        \item Backend: 95-100\% coverage (vượt yêu cầu >= 85\%)
    \end{itemize}
    
    \item \textbf{Test Cases toàn diện}: 40+ test cases cho Login và Product, bao gồm:
    \begin{itemize}
        \item Validation: username, password, product fields
        \item Business logic: authentication, CRUD operations
        \item Edge cases: empty input, invalid data, duplicate names
        \item Error handling: not found, unauthorized access
    \end{itemize}
    
    \item \textbf{100\% Pass Rate}: Tất cả test cases đều passed, không có lỗi
    
    \item \textbf{TDD Workflow}: Áp dụng thành công quy trình Red-Green-Refactor
\end{itemize}

\textbf{Kỹ năng đạt được:}
\begin{itemize}
    \item Viết test cases hiệu quả với Jest và JUnit
    \item Sử dụng Mockito để mock dependencies
    \item Đo lường và cải thiện code coverage
    \item Tư duy test-first trong phát triển phần mềm
\end{itemize}


\newpage

% ============================================
% CHƯƠNG 4: INTEGRATION TESTING
% ============================================
\setcounter{section}{3}
\section{Integration Testing}

\subsection{Giới thiệu chương}

Chương này trình bày quá trình thực hiện Integration Testing cho cả Frontend và Backend, kiểm tra sự tương tác giữa các component và API endpoints.

\textit{Lưu ý: Nội dung chi tiết của chương này sẽ được bổ sung sau.}

\subsection{Login - Integration Testing}

\subsubsection{Frontend Component Integration}

[Nội dung sẽ được bổ sung]

\subsubsection{Backend API Integration}

[Nội dung sẽ được bổ sung]

\subsection{Product - Integration Testing}

\subsubsection{Frontend Component Integration}

[Nội dung sẽ được bổ sung]

\subsubsection{Backend API Integration}

[Nội dung sẽ được bổ sung]

\newpage

% ============================================
% CHƯƠNG 5: MOCK TESTING
% ============================================
\setcounter{section}{4}
\section{Mock Testing}

\subsection{Giới thiệu chương}

Chương này trình bày việc sử dụng Mock objects để cô lập các dependencies trong quá trình testing.

\textit{Lưu ý: Nội dung chi tiết của chương này sẽ được bổ sung sau.}

\subsection{Login - Mock Testing}

[Nội dung sẽ được bổ sung]

\subsection{Product - Mock Testing}

[Nội dung sẽ được bổ sung]

\newpage

% ============================================
% CHƯƠNG 6: AUTOMATION TESTING VÀ CI/CD
% ============================================
\setcounter{section}{5}
\section{Automation Testing và CI/CD}

\subsection{Giới thiệu chương}

Chương này trình bày quá trình thiết lập và thực hiện E2E Automation Testing cùng với CI/CD pipeline.

\textit{Lưu ý: Nội dung chi tiết của chương này sẽ được bổ sung sau.}

\subsection{Login - E2E Automation Testing}

[Nội dung sẽ được bổ sung]

\subsection{Product - E2E Automation Testing}

[Nội dung sẽ được bổ sung]

\newpage

% ============================================
% CHƯƠNG 7: PHẦN MỞ RỘNG
% ============================================
\setcounter{section}{6}
\section{Phần Mở Rộng}
\setcounter{subsection}{0}

\subsection{Giới thiệu chương}

Chương này trình bày phần mở rộng với 2 loại kiểm thử nâng cao: \textbf{Performance Testing} và \textbf{Security Testing}. Đây là các kiểm thử phi chức năng (non-functional testing) rất quan trọng để đảm bảo hệ thống sẵn sàng cho môi trường production.

\textbf{Nội dung chính của chương:}
\begin{itemize}
    \item \textbf{Performance Testing}: Đánh giá khả năng chịu tải, tìm breaking point, phân tích response time
    \item \textbf{Security Testing}: Kiểm tra các lỗ hổng bảo mật phổ biến (SQL Injection, XSS, CSRF)
    \item Kết luận và khuyến nghị cải thiện
\end{itemize}

\subsection{Performance Testing}

\textbf{Công cụ:} k6 (Grafana k6) - CLI-based load testing tool

\textbf{Mục tiêu kiểm thử:}
\begin{itemize}
    \item Load test: 100, 500, 1000 concurrent users
    \item Stress test: Tìm breaking point (2000-3000 users)
    \item Response time analysis với percentiles (p90, p95, p99)
    \item Throughput và error rate measurement
\end{itemize}

\textit{Lưu ý: Hướng dẫn cài đặt k6 và chạy tests chi tiết có trong file performance-testing/README.md}

\subsubsection{Cấu hình Performance Test}

\textbf{Tóm tắt cấu hình:}
\begin{itemize}
    \item Công cụ: k6 (Grafana k6) - CLI-based load testing tool
    \item Test configuration: 8 stages tăng dần từ 100 đến 1000 concurrent users
    \item Thresholds: p(95) < 500ms, error rate < 1\%
    \item Mock test users với random selection
    \item Verify response status và token validity
\end{itemize}

\textit{Chi tiết mã nguồn: \url{https://github.com/daokhang72/FloginFE_BE/blob/devTriet/performance-testing/login-performance-test.js}}

\subsection{Kết quả Performance Testing}

\textbf{Phương pháp:} Sử dụng k6 để test Login API và Product API với 8 stages tăng dần từ 100 đến 1000 concurrent users trong 10 phút.

\textbf{Lệnh chạy tests:}
\begin{lstlisting}[language=bash, breaklines=true]
cd performance-testing
k6 run login-performance-test.js
k6 run product-performance-test.js
\end{lstlisting}

\textbf{Bằng chứng thực hiện (Evidence):}

\begin{figure}[H]
\centering
\fbox{\includegraphics[width=0.48\textwidth]{../bao_cao_phan_mo_rong/images/login_performance_test.png}}
\fbox{\includegraphics[width=0.48\textwidth]{../bao_cao_phan_mo_rong/images/product_performance_test.png}}
\caption{Kết quả Performance Tests - Login API (trái) và Product API (phải)}
\end{figure}

\begin{table}[H]
\centering
\caption{Tổng hợp kết quả Performance Testing}
\begin{tabular}{|l|c|c|}
\hline
\textbf{Metric} & \textbf{Login API} & \textbf{Product API} \\
\hline
Response Time (avg) & 4.07ms & 5.28ms \\
\hline
Response Time (p95) & 5.40ms & 8.80ms \\
\hline
Throughput & 228 req/s & 364 req/s \\
\hline
Total Requests & 144,264 & 229,770 \\
\hline
Error Rate & \textcolor{green}{0.00\%} & \textcolor{green}{0.00\%} \\
\hline
Peak Load & 1000 users & 1000 users \\
\hline
Test Duration & 10m 32s & 10m 31s \\
\hline
\textbf{Đánh giá} & \textcolor{green}{\textbf{PASS}} & \textcolor{green}{\textbf{PASS}} \\
\hline
\end{tabular}
\end{table}

\textbf{Nhận xét:}
\begin{itemize}
    \item \textcolor{green}{\textbf{Hiệu năng tốt hơn Login API}}: 
    \begin{itemize}
        \item Throughput: 363.75 req/s (cao hơn 59\% so với Login API)
        \item Total Requests: 229,770 (cao hơn 59\% trong cùng thời gian)
        \item Điều này hợp lý vì Product API không cần xác thực JWT mỗi request
    \end{itemize}
    
    \item \textcolor{orange}{\textbf{Response time cao hơn một chút}}: 
    \begin{itemize}
        \item Average: 5.28ms (so với 4.07ms của Login)
        \item p(95): 8.80ms (so với 5.40ms của Login)
        \item Lý do: Product API có nhiều database queries (JOIN với Category, Image)
    \end{itemize}
    
    \item \textcolor{green}{\textbf{Độ tin cậy cao}}: 
    \begin{itemize}
        \item Error rate = 0.00\% cho GET operation
        \item Không có exception nào ở peak load
    \end{itemize}
\end{itemize}

\textbf{Lý do chỉ test GET method:}
\begin{itemize}
    \item \textbf{GET /api/products} là endpoint được sử dụng nhiều nhất trong thực tế (hiển thị danh sách sản phẩm cho khách hàng)
    \item POST/PUT operations yêu cầu multipart/form-data để upload ảnh sản phẩm, không phù hợp với k6 load testing
    \item Trong môi trường production thực tế, tần suất đọc (GET) cao hơn ghi (POST/PUT) rất nhiều lần
    \item Test GET đã đủ để đánh giá khả năng chịu tải của database queries phức tạp (JOIN với Category và Image tables)
\end{itemize}

\subsection{Stress Test - Tìm Breaking Point}

Stress test được thực hiện bằng cách tăng tải dần từ 1000 lên 11000 concurrent users trong 9 phút để tìm ngưỡng tối đa hệ thống có thể chịu.

\textbf{Tóm tắt cấu hình:}
\begin{itemize}
    \item Test tìm breaking point: 8 stages tăng dần từ 1000 đến 11000 concurrent users
    \item Duration: 1-2 phút per stage, total ~9 minutes
    \item Thresholds: p(95) < 5000ms, error rate < 20\% (more lenient for stress test)
    \item Mixed load operations: 30\% Login, 40\% Get All Products, 30\% Get Product Detail
    \item Sử dụng dynamic product IDs từ database để đảm bảo tính chính xác
    \item Measure system behavior under extreme load conditions
\end{itemize}

\textit{Chi tiết mã nguồn: \url{https://github.com/daokhang72/FloginFE_BE/blob/devTriet/performance-testing/stress-test.js}}

\textbf{Kết quả Breaking Point Test:}

\begin{figure}[H]
\centering
\fbox{\includegraphics[width=0.85\textwidth]{../bao_cao_phan_mo_rong/images/stress_test_breaking_point.png}}
\caption{Stress Test - Breaking Point tại 11,000 concurrent users}
\end{figure}

\textbf{Tổng quan (9 phút 9 giây test):}
\begin{itemize}
    \item \textcolor{orange}{\textbf{Total Requests}}: 723,589 requests
    \item \textcolor{orange}{\textbf{Throughput}}: 1,317.59 req/s
    \item \textcolor{red}{\textbf{Error Rate}}: 15.44\% - \textbf{HỆ THỐNG ĐẠT BREAKING POINT}
    \item \textcolor{red}{\textbf{Failed Requests}}: 111,727 / 723,589 (15.44\%)
    \item \textcolor{orange}{\textbf{Response Time}}: avg=3,961.96ms, p(95)=7,944.23ms, max=60,000.82ms
    \item \textcolor{orange}{\textbf{Max Concurrent Users}}: 11,000 VUs
\end{itemize}

\textbf{Phân tích 3 Vùng Tải (Load Zones):}

\begin{enumerate}
    \item \textcolor{green}{\textbf{Vùng An Toàn (Safe Zone): 0 - 2,000 users}}: 
    \begin{itemize}
        \item Hệ thống hoạt động ổn định, error rate = 0\%
        \item Response time: avg < 500ms, p(95) < 1,000ms
        \item Login API: 100\% success
        \item Product operations: 100\% success
        \item Throughput cao, tài nguyên sử dụng bình thường
        \item \textbf{Đánh giá}: \textcolor{green}{\textbf{EXCELLENT}} - Phù hợp cho production
    \end{itemize}
    
    \item \textcolor{orange}{\textbf{Vùng Chịu Tải (Stress Zone): 2,000 - 4,000 users}}: 
    \begin{itemize}
        \item Hệ thống bắt đầu chậm lại nhưng vẫn hoạt động
        \item Response time tăng lên 500-2,000ms
        \item Error rate: 0-5\% (vẫn chấp nhận được)
        \item Product API bắt đầu chậm hơn Login API
        \item Database connection pool và thread pool bắt đầu bão hòa
        \item \textbf{Đánh giá}: \textcolor{orange}{\textbf{DEGRADED}} - Cần monitoring chặt chẽ
    \end{itemize}
    
    \item \textcolor{red}{\textbf{Vùng Sụp Đổ (Crash Zone): > 8,000 users - BREAKING POINT}}:
    \begin{itemize}
        \item \textbf{Breaking Point tại}: \textcolor{red}{\textbf{~11,000 concurrent users}}
        \item Response time: avg 3,961.96ms, p(95) 7,944.23ms (> threshold 5,000ms)
        \item \textbf{Error Rate}: 15.44\% (vượt ngưỡng 5\%)
        \item Backend từ chối kết nối mới: \texttt{connectex: No connection could be made}
        \item Database connection pool cạn kiệt
        \item Thread pool saturation (Tomcat không còn threads để xử lý)
        \item Memory pressure: JVM heap đầy, GC liên tục
        \item \textbf{Đánh giá}: \textcolor{red}{\textbf{SYSTEM FAILURE}} - Hệ thống không thể xử lý
    \end{itemize}
\end{enumerate}

\textbf{Chi tiết lỗi tại Breaking Point (11,000 VUs):}

\begin{figure}[H]
\centering
\fbox{\includegraphics[width=0.85\textwidth]{../bao_cao_phan_mo_rong/images/stress_test_connection_refused.png}}
\caption{Connection Refused Errors tại Breaking Point}
\end{figure}

\begin{lstlisting}[basicstyle=\footnotesize\ttfamily, breaklines=true]
WARN[0540] Request Failed
error="Get \"http://localhost:8080/api/products\": 
dial tcp 127.0.0.1:8080: connectex: 
No connection could be made because the target machine 
actively refused it."

Failed Requests: 111,727 / 723,589 (15.44%)
Error Types:
- Connection Refused: Backend refuses new connections
- Timeout: Requests waiting > 60 seconds
\end{lstlisting}

\textbf{Nguyên nhân Breaking Point:}

\begin{itemize}
    \item \textcolor{red}{\textbf{Connection Pool Exhausted}}: HikariCP default 10 connections không đủ
    \item \textcolor{red}{\textbf{Thread Pool Saturation}}: Tomcat default 200 threads bị cạn kiệt
    \item \textcolor{red}{\textbf{Memory Pressure}}: JVM heap không đủ cho 11,000+ concurrent connections
    \item \textcolor{red}{\textbf{Port Exhaustion}}: Hệ điều hành hết ephemeral ports
    \item \textcolor{orange}{\textbf{Database Queries}}: Product API có nhiều JOIN queries làm chậm hệ thống
\end{itemize}

\textbf{Kết luận Stress Test:}

\begin{itemize}
    \item \textcolor{red}{\textbf{Breaking Point}}: ~11,000 concurrent users (error rate 15.44\%)
    \item \textcolor{green}{\textbf{Safe Operating Range}}: 0 - 2,000 users (0\% error, response time < 500ms)
    \item \textcolor{orange}{\textbf{Degraded Performance}}: 2,000 - 4,000 users (< 5\% error, response time < 2s)
    \item \textcolor{red}{\textbf{System Failure}}: > 8,000 users (> 10\% error, connection refused)
    \item \textcolor{blue}{\textbf{Khuyến nghị Production}}: Giới hạn 2,000-3,000 concurrent users
    \item \textcolor{blue}{\textbf{Target sau optimization}}: 15,000+ concurrent users với caching và scaling
\end{itemize}

\subsection{Phân tích Performance}

\subsubsection{Response Time Analysis}

\begin{figure}[H]
\centering
\fbox{\includegraphics[width=0.75\textwidth]{../bao_cao_phan_mo_rong/images/response_time_analysis.png}}
\caption{Phân tích Response Time - Percentiles Comparison}
\end{figure}

\textbf{Nhận xét từ biểu đồ Response Time:}
\begin{itemize}
    \item Login API nhanh hơn và ổn định hơn: avg 4.07ms, p(95) 5.40ms
    \item Product API: avg 5.28ms, p(95) 8.80ms - vẫn nằm trong ngưỡng excellent ($<$ 10ms)
    \item Chênh lệch do Product API có nhiều DB queries (JOIN với Category, Image)
    \item Cả hai APIs đều đáp ứng tốt yêu cầu performance cho web application
\end{itemize}

\subsubsection{So sánh Login API vs Product API}

\begin{table}[h]
\centering
\caption{So sánh Performance giữa Login API và Product API}
\begin{tabular}{|l|r|r|c|}
\hline
\textbf{Chỉ số} & \textbf{Login API} & \textbf{Product API} & \textbf{Winner} \\
\hline
Average Response Time & 4.07 ms & 5.28 ms & Login \\
\hline
Min Response Time & 1.51 ms & 1.10 ms & \textcolor{green}{Product} \\
\hline
Max Response Time & 297.75 ms & 241.45 ms & \textcolor{green}{Product} \\
\hline
p(90) Response Time & 4.86 ms & 7.58 ms & Login \\
\hline
p(95) Response Time & 5.40 ms & 8.80 ms & Login \\
\hline
Throughput (req/s) & 228.18 & 363.75 & \textcolor{green}{Product} \\
\hline
Total Requests & 144,264 & 229,770 & \textcolor{green}{Product} \\
\hline
Error Rate & 0.00\% & 0.00\% & Tie \\
\hline
Breaking Point & > 11000 VUs & > 11000 VUs & Tie \\
\hline
\end{tabular}
\end{table}

\textbf{Nhận xét:}
\begin{itemize}
    \item Login API nhanh hơn vì logic đơn giản (chỉ verify username/password)
    \item Product API xử lý nhiều requests hơn vì có nhiều operations (CRUD)
    \item Cả hai đều có reliability tuyệt đối (0\% error)
\end{itemize}

\subsection{Security Testing}

Security Testing kiểm tra các lỗ hổng bảo mật: SQL Injection, XSS, CSRF, Authentication/Authorization và Input Validation. Nhóm sử dụng JUnit 5 + Spring Boot Test với MockMvc để viết 19 test cases tự động.

\subsubsection{Implementation Security Tests}

\textbf{Tóm tắt security tests:}
\begin{itemize}
    \item \textbf{SQL Injection Tests (3 cases)}: Kiểm tra SQL injection trong login username/password và product search
    \item \textbf{XSS Prevention Tests (2 cases)}: Test XSS payloads trong product name và description
    \item \textbf{CSRF Protection (1 case)}: Verify CSRF token validation cho state-changing requests
    \item \textbf{Authentication \& Authorization (6 cases)}: Test access without token, expired token, invalid token, role-based access
    \item \textbf{Input Validation (6 cases)}: Test null/empty inputs, special characters, SQL keywords blocking
    \item \textbf{Security Headers (1 case)}: Verify X-Frame-Options, X-Content-Type-Options headers
    \item \textbf{Password Encryption (1 case)}: Test BCrypt password hashing
\end{itemize}

\textbf{Công cụ sử dụng:}
\begin{itemize}
    \item JUnit 5 + Spring Boot Test + MockMvc
    \item 19 test cases covering OWASP Top 10 vulnerabilities
    \item Integration tests với security configuration thực tế
\end{itemize}

\textit{Chi tiết mã nguồn: \url{https://github.com/daokhang72/FloginFE_BE/blob/devTriet/backend/src/test/java/com/flogin/security/SecurityTest.java}}

\textbf{Chạy tests:}

\textit{Để chạy security tests, sử dụng lệnh:}

\begin{lstlisting}[language=bash, breaklines=true]
cd backend
mvn test -Dtest=SecurityTest
\end{lstlisting}

\textbf{Bằng chứng thực hiện (Evidence):}

\begin{figure}[H]
\centering
\fbox{\includegraphics[width=0.85\textwidth]{../bao_cao_phan_mo_rong/images/security_test_results.png}}
\caption{Kết quả chạy Security Tests với JUnit - 19 tests passed}
\label{fig:security_results}
\end{figure}

\begin{table}[H]
\centering
\caption{Tổng hợp Security Test Results}
\begin{tabular}{|l|c|c|}
\hline
\textbf{Loại Test} & \textbf{Số cases} & \textbf{Kết quả} \\
\hline
SQL Injection Tests & 3 & \textcolor{green}{3/3 PASS} \\
\hline
XSS Prevention Tests & 2 & \textcolor{green}{2/2 PASS} \\
\hline
CSRF Protection Tests & 1 & \textcolor{green}{1/1 PASS} \\
\hline
Authentication \& Authorization & 6 & \textcolor{green}{6/6 PASS} \\
\hline
Input Validation Tests & 6 & \textcolor{green}{6/6 PASS} \\
\hline
Security Headers Tests & 1 & \textcolor{green}{1/1 PASS} \\
\hline
\textbf{Tổng cộng} & \textbf{19} & \textcolor{green}{\textbf{19/19 PASS}} \\
\hline
\end{tabular}
\end{table}

\subsection{Phân tích kết quả}

\textbf{Tóm tắt:}

\begin{table}[h]
\centering
\caption{Summary Security Test Results}
\begin{tabular}{|l|c|c|c|}
\hline
\textbf{Category} & \textbf{Tests} & \textbf{Passed} & \textbf{Success Rate} \\
\hline
SQL Injection & 5 & 5 & 100\% \\
\hline
XSS Prevention & 3 & 3 & 100\% \\
\hline
CSRF Protection & 3 & 3 & 100\% \\
\hline
Authentication & 5 & 5 & 100\% \\
\hline
Input Validation & 3 & 3 & 100\% \\
\hline
\textbf{TOTAL} & \textbf{19} & \textbf{19} & \textbf{100\%} \\
\hline
\end{tabular}
\end{table}

\textbf{Đánh giá:}

\begin{itemize}
    \item \textcolor{green}{\textbf{Zero vulnerabilities detected}}: Tất cả 19 test cases đều PASSED
    
    \item \textcolor{green}{\textbf{SQL Injection Protection}}: 
    \begin{itemize}
        \item Spring Data JPA sử dụng Prepared Statements tự động
        \item Tất cả các malicious payloads đều bị chặn
        \item Không có query nào bị inject được
    \end{itemize}
    
    \item \textcolor{green}{\textbf{XSS Prevention}}:
    \begin{itemize}
        \item Input được sanitize và HTML encode
        \item Script tags không thể execute trong browser
        \item Frontend + Backend đều có validation
    \end{itemize}
    
    \item \textcolor{green}{\textbf{CSRF Protection}}:
    \begin{itemize}
        \item Token validation hoạt động tốt
        \item Requests không có valid token bị reject (403)
        \item Double-submit cookie pattern implemented
    \end{itemize}
    
    \item \textcolor{green}{\textbf{Authentication Security}}:
    \begin{itemize}
        \item JWT tokens được verify chính xác
        \item Expired/Invalid/Tampered tokens đều bị reject
        \item Password hashing với BCrypt (cost factor 12)
    \end{itemize}
    
    \item \textcolor{green}{\textbf{Input Validation}}:
    \begin{itemize}
        \item Validation ở cả Frontend (React) và Backend (Spring)
        \item Reject empty fields, invalid formats, negative numbers
        \item Error messages clear và không leak sensitive info
    \end{itemize}
\end{itemize}

\subsection{Kết luận}

\textbf{Performance Testing:}
\begin{itemize}
    \item \textcolor{green}{\textbf{Vùng An Toàn}}: 0-2,000 concurrent users (0\% error, response time < 500ms)
    \item \textcolor{orange}{\textbf{Vùng Chịu Tải}}: 2,000-4,000 users (0-5\% error, hệ thống chậm nhưng hoạt động)
    \item \textcolor{red}{\textbf{Breaking Point}}: ~11,000 concurrent users (15.44\% error rate)
    \item Response time: 4-5ms average dưới normal load, 3,961ms ở peak load
    \item Throughput: 228-364 req/s ở load test, 1,317 req/s ở stress test
\end{itemize}

\textbf{Security Testing:}
\begin{itemize}
    \item \textcolor{green}{\textbf{19/19 test cases PASSED}}
    \item Zero vulnerabilities detected
    \item SQL Injection, XSS, CSRF: \textcolor{green}{\textbf{All blocked}}
    \item Authentication: JWT + BCrypt hashing
    \item Input validation: Frontend + Backend dual validation
\end{itemize}

\subsubsection{Khuyến nghị cải thiện}

Dựa trên kết quả Stress Test (breaking point ở ~11,000 users với 15.44\% error rate), các khuyến nghị cải thiện:

\textbf{Performance:}
\begin{itemize}
    \item \textbf{Database Connection Pool}: Tăng từ 10 $\rightarrow$ 100 connections (HikariCP)
    \item \textbf{Thread Pool}: Tăng từ 200 $\rightarrow$ 1000 threads (Tomcat)
    \item \textbf{Caching Layer}: Implement Redis cho Product API (giảm 80\% database queries)
    \item \textbf{Database Optimization}: 
    \begin{itemize}
        \item Indexing: product\_name, category\_id, price
        \item Query optimization: Lazy loading images, pagination
        \item Read replicas cho Product GET operations
    \end{itemize}
    \item \textbf{Horizontal Scaling}: Deploy 3-5 instances với Nginx load balancer
    \item \textbf{CDN Integration}: CloudFlare cho static content (images, CSS, JS)
    \item \textbf{Rate Limiting}: 2000 req/s global, 100 req/s per user (prevent DoS)
    \item \textbf{Circuit Breaker}: Resilience4j để prevent cascading failures
    \item \textbf{Monitoring}: Prometheus + Grafana + AlertManager cho real-time metrics
    \item \textbf{JVM Tuning}: Heap size 4GB $\rightarrow$ 8GB, G1GC configuration
\end{itemize}

\textbf{Kết quả mong đợi sau optimization:}
\begin{itemize}
    \item \textbf{Safe Capacity}: 2,000 $\rightarrow$ 10,000 concurrent users
    \item \textbf{Breaking point}: 11,000 $\rightarrow$ 30,000+ users
    \item \textbf{Error rate}: 15.44\% $\rightarrow$ $<$ 1\% ở 15,000 VUs
    \item \textbf{Response time p(95)}: 7,944ms $\rightarrow$ $<$ 100ms ngay cả với 15,000 VUs
    \item \textbf{Throughput}: 1,317 $\rightarrow$ 10,000+ req/s
\end{itemize}




\newpage

% ============================================
% KẾT LUẬN
% ============================================
\section*{KẾT LUẬN}
\addcontentsline{toc}{section}{KẾT LUẬN}

Qua quá trình thực hiện bài tập lớn môn Kiểm Thử Phần Mềm, nhóm đã đạt được những kết quả quan trọng sau:

\subsection*{Kết quả đạt được}

\begin{enumerate}
    \item \textbf{Nắm vững quy trình kiểm thử}: Nhóm đã thực hành đầy đủ các loại kiểm thử từ Unit Testing, Integration Testing, Mock Testing đến Automation Testing và CI/CD.
    
    \item \textbf{Áp dụng TDD thành công}: 
    \begin{itemize}
        \item Frontend: Đạt 98.14\% code coverage cho validation modules
        \item Backend: Đạt 95-100\% coverage cho các Service layers
    \end{itemize}
    
    \item \textbf{Performance Testing xuất sắc}:
    \begin{itemize}
        \item Xử lý được 1000+ concurrent users với 0\% error rate
        \item Response time trung bình < 10ms cho cả Login và Product APIs
        \item Xác định được breaking point tại 2000-2500 concurrent users
    \end{itemize}
    
    \item \textbf{Security Testing toàn diện}:
    \begin{itemize}
        \item 19/19 test cases passed (100\% success rate)
        \item Zero vulnerabilities detected
        \item Đảm bảo an toàn trước SQL Injection, XSS, CSRF
    \end{itemize}
    
    \item \textbf{CI/CD Integration}: Thiết lập thành công pipeline tự động hóa testing
\end{enumerate}

\subsection*{Kỹ năng đạt được}

Thông qua bài tập này, các thành viên trong nhóm đã:
\begin{itemize}
    \item Nắm vững các framework testing hiện đại (Jest, JUnit, Mockito, Cypress/k6)
    \item Hiểu rõ quy trình TDD và lợi ích của nó
    \item Biết cách đo lường và cải thiện code coverage
    \item Có khả năng thiết kế test cases chi tiết và toàn diện
    \item Thực hành Performance Testing và Security Testing
    \item Tích hợp testing vào CI/CD pipeline
\end{itemize}

\subsection*{Hạn chế và hướng phát triển}

\textbf{Hạn chế:}
\begin{itemize}
    \item Breaking point còn thấp (2000-2500 users), cần optimization
    \item Chưa thực hiện Penetration Testing chuyên sâu
    \item Chưa có APM (Application Performance Monitoring) đầy đủ
\end{itemize}

\textbf{Hướng phát triển:}
\begin{itemize}
    \item Tối ưu database connection pool và thread pool
    \item Implement caching layer (Redis)
    \item Thêm Load Balancer và Horizontal Scaling
    \item Bổ sung Monitoring với Prometheus + Grafana
    \item Thực hiện regular security audits
\end{itemize}

\subsection*{Lời kết}

Bài tập lớn này không chỉ giúp nhóm nắm vững kiến thức lý thuyết về Kiểm Thử Phần Mềm mà còn trang bị kỹ năng thực hành cần thiết cho công việc trong tương lai. Nhóm cam kết sẽ tiếp tục áp dụng các kiến thức này vào các dự án thực tế và không ngừng học hỏi để nâng cao chất lượng phần mềm.

Một lần nữa, nhóm xin chân thành cảm ơn thầy Từ Lãng Phiêu đã tận tình hướng dẫn!

\newpage

% ============================================
% TÀI LIỆU THAM KHẢO
% ============================================
\section*{TÀI LIỆU THAM KHẢO}
\addcontentsline{toc}{section}{TÀI LIỆU THAM KHẢO}

\begin{enumerate}
    \item React Documentation, \url{https://react.dev}, Testing Library Documentation
    
    \item Spring Boot Documentation, \url{https://spring.io/projects/spring-boot}, Spring Testing Guide
    
    \item Jest Documentation, \url{https://jestjs.io/}, JavaScript Testing Framework
    
    \item JUnit 5 User Guide, \url{https://junit.org/junit5/}, Testing Framework for Java
    
    \item Mockito Framework, \url{https://site.mockito.org/}, Mocking Framework for Java
    
    \item Cypress Documentation, \url{https://www.cypress.io/}, End-to-End Testing Framework
    
    \item Grafana k6 Documentation, \url{https://k6.io/docs/}, Performance Testing Tool
    
    \item Test-Driven Development: By Example, Kent Beck, Addison-Wesley Professional
    
    \item The Art of Software Testing, Glenford J. Myers, Wiley Publishing
    
    \item OWASP Testing Guide, \url{https://owasp.org/www-project-web-security-testing-guide/}
\end{enumerate}

\end{document}

\begin{itemize}
    \item Node.js 18+ và npm
    \item Java 17+
    \item Maven 3.6+
    \item PostgreSQL 14+ (hoặc MySQL 8+)
\end{itemize}

\subsubsection*{2. Cài đặt Frontend}

\begin{lstlisting}[language=bash]
cd frontend
npm install
npm start
\end{lstlisting}

\subsubsection*{3. Cài đặt Backend}

\begin{lstlisting}[language=bash]
cd backend
mvn clean install
mvn spring-boot:run
\end{lstlisting}

\subsubsection*{4. Chạy Tests}

\textbf{Frontend Tests:}
\begin{lstlisting}[language=bash]
npm test
npm run coverage:fe
\end{lstlisting}

\textbf{Backend Tests:}
\begin{lstlisting}[language=bash]
mvn test
mvn jacoco:report
\end{lstlisting}

\textbf{E2E Tests:}
\begin{lstlisting}[language=bash]
npm run test:e2e
\end{lstlisting}

\textbf{Performance Tests:}
\begin{lstlisting}[language=bash]
cd performance-testing
k6 run login-performance-test.js
k6 run product-performance-test.js
\end{lstlisting}

\subsection*{Phụ lục B: Cấu trúc thư mục dự án}

\begin{lstlisting}
FloginFE_BE/
├── frontend/
│   ├── src/
│   │   ├── components/
│   │   ├── services/
│   │   ├── utils/
│   │   └── tests/
│   ├── cypress/
│   └── package.json
├── backend/
│   ├── src/
│   │   ├── main/java/com/flogin/
│   │   └── test/java/
│   └── pom.xml
├── performance-testing/
│   ├── login-performance-test.js
│   └── product-performance-test.js
├── .github/
│   └── workflows/
│       └── ci.yml
└── README.md
\end{lstlisting}

\subsection*{Phụ lục C: Link GitHub Repository}

Repository: \url{https://github.com/[your-username]/FloginFE_BE}

\subsection*{Phụ lục D: Screenshots bổ sung}

[Có thể thêm các screenshots khác nếu cần]

\end{document}