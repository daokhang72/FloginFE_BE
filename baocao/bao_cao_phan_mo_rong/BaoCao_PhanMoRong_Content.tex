\section{Phần Mở Rộng}
\setcounter{subsection}{0}

\noindent
Phần này thực hiện 2 loại kiểm thử nâng cao: \textbf{Performance Testing} để đánh giá khả năng chịu tải và hiệu năng của hệ thống, và \textbf{Security Testing} để kiểm tra các lỗ hổng bảo mật. Cả hai đều là yêu cầu quan trọng trong phát triển phần mềm chuyên nghiệp.

\subsection{Performance Testing}

\subsubsection{Yêu cầu và Mục tiêu}

Theo yêu cầu của đề bài tập lớn, nhóm cần thực hiện:

\begin{enumerate}
    \item Setup công cụ kiểm thử hiệu năng (JMeter hoặc k6)
    \item Viết performance tests cho Login API:
    \begin{itemize}
        \item Load test: 100, 500, 1000 concurrent users
        \item Stress test: Tìm breaking point
        \item Response time analysis
    \end{itemize}
    \item Viết performance tests cho Product API
    \item Phân tích kết quả và đưa ra recommendations
\end{enumerate}

\subsubsection{Công cụ sử dụng}

Nhóm đã chọn \textbf{k6} (Grafana k6) làm công cụ kiểm thử hiệu năng vì:

\begin{itemize}
    \item \textbf{Hiện đại và Developer-friendly}: Viết test bằng JavaScript (ES6+), dễ tích hợp với codebase hiện có
    \item \textbf{CLI-based}: Chạy trực tiếp từ terminal, không cần GUI phức tạp như JMeter
    \item \textbf{Cloud-ready}: Hỗ trợ xuất kết quả sang JSON, dễ tích hợp CI/CD
    \item \textbf{Hiệu suất cao}: Viết bằng Go, xử lý được hàng nghìn concurrent users
    \item \textbf{Thống kê chi tiết}: Cung cấp percentiles (p90, p95, p99), throughput, error rate
\end{itemize}

\clearpage

\textbf{Cài đặt k6:}

\begin{lstlisting}[language=bash]
# Windows (using Chocolatey)
choco install k6

# macOS (using Homebrew)
brew install k6

# Linux
sudo gpg -k
sudo gpg --no-default-keyring --keyring /usr/share/keyrings/k6-archive-keyring.gpg --keyserver hkp://keyserver.ubuntu.com:80 --recv-keys C5AD17C747E3415A3642D57D77C6C491D6AC1D69
echo "deb [signed-by=/usr/share/keyrings/k6-archive-keyring.gpg] https://dl.k6.io/deb stable main" | sudo tee /etc/apt/sources.list.d/k6.list
sudo apt-get update
sudo apt-get install k6
\end{lstlisting}

\subsection{Performance Tests cho Login API}

\subsubsection{Thiết kế Test Scenarios}

Login API là endpoint quan trọng nhất của hệ thống, xử lý xác thực người dùng. Nhóm thiết kế 8 stages để mô phỏng tải tăng dần và giảm dần, từ 100 VUs (Virtual Users) khởi động cho đến 1000 VUs ở peak load. Mục tiêu là đánh giá khả năng chịu tải, thời gian phản hồi và độ ổn định của authentication service trong điều kiện tải cao.

\begin{table}[ht]
\centering
\caption{Load Test Stages cho Login API}
\begin{tabular}{|c|c|c|l|}
\hline
\textbf{Stage} & \textbf{Duration} & \textbf{Target VUs} & \textbf{Mục đích} \\
\hline
1 & 1m & 100 & Warm-up, khởi động hệ thống \\
\hline
2 & 1m & 100 & Baseline measurement \\
\hline
3 & 1m & 300 & Tăng tải lên 3x \\
\hline
4 & 2m & 500 & Load test trung bình \\
\hline
5 & 2m & 800 & Load test cao \\
\hline
6 & 2m & 1000 & Stress test - tìm breaking point \\
\hline
7 & 1m & 500 & Recovery test \\
\hline
8 & 30s & 0 & Cool down, kết thúc \\
\hline
\end{tabular}
\end{table}

\subsubsection{Kết quả thực thi}

\textit{Để chạy test, sử dụng lệnh:}

\begin{lstlisting}[language=bash]
cd performance-testing
k6 run login-performance-test.js
\end{lstlisting}

\textbf{Bằng chứng thực hiện (Evidence):}

\begin{figure}[h]
\centering
\fbox{\includegraphics[width=0.95\textwidth]{../bao_cao_phan_mo_rong/images/login_performance_test.png}}
\caption{Kết quả Performance Test - Login API (k6 output từ Terminal)}
\label{fig:login_perf}
\end{figure}

\subsubsection{Phân tích kết quả Login API}

\textbf{Tóm tắt các chỉ số quan trọng:}

\begin{itemize}
    \item \textbf{Response Time:} avg = 4.07ms, min = 1.51ms, max = 297.75ms
    \item \textbf{Percentiles:} p(90) = 4.86ms, p(95) = 5.40ms
    \item \textbf{Throughput:} 228.18 req/s, Total = 144,264 requests
    \item \textbf{Error Rate:} 0.00\% (100\% success)
    \item \textbf{Duration:} 10m 32.2s với 144,261 completed iterations
\end{itemize}

\textbf{Đánh giá chi tiết:}

\begin{itemize}
    \item \textcolor{green}{\textbf{Thời gian phản hồi xuất sắc}}: 
    \begin{itemize}
        \item Average 4.07ms là rất tốt cho Authentication API
        \item p(95) = 5.40ms nghĩa là 95\% requests hoàn thành dưới 5.5ms
        \item Maximum 297.75ms chỉ xảy ra ở thời điểm peak load (1000 VUs)
    \end{itemize}
    
    \item \textcolor{green}{\textbf{Throughput ổn định}}: 
    \begin{itemize}
        \item 228.18 req/s là con số tốt cho 1000 concurrent users
        \item Server xử lý được 144,264 requests trong 10m 32s
    \end{itemize}
    
    \item \textcolor{green}{\textbf{Độ tin cậy hoàn hảo}}: 
    \begin{itemize}
        \item Error rate = 0.00\% nghĩa là không có request nào thất bại
        \item Hệ thống ổn định ngay cả ở peak load
    \end{itemize}
\end{itemize}

\clearpage

\subsection{Performance Tests cho Product API}

\subsubsection{Thiết kế Test Scenarios}

Product API test sử dụng cùng cấu trúc 8 stages, nhưng bao gồm nhiều operations:

\begin{itemize}
    \item \textbf{READ Operations} (70\%):
    \begin{itemize}
        \item GET /api/products (List all)
        \item GET /api/products/\{id\} (Get by ID)
    \end{itemize}
    
    \item \textbf{WRITE Operations} (30\%):
    \begin{itemize}
        \item POST /api/products (Create)
        \item PUT /api/products/\{id\} (Update)
        \item DELETE /api/products/\{id\} (Delete)
    \end{itemize}
\end{itemize}

Tỷ lệ 70-30 mô phỏng thực tế: người dùng thường xem sản phẩm nhiều hơn là thêm/sửa/xóa.

\subsubsection{Kết quả thực thi}

\textit{Để chạy test, sử dụng lệnh:}

\begin{lstlisting}[language=bash]
cd performance-testing
k6 run product-performance-test.js
\end{lstlisting}

\textbf{Bằng chứng thực hiện (Evidence):}

\begin{figure}[h]
\centering
\fbox{\includegraphics[width=0.95\textwidth]{../bao_cao_phan_mo_rong/images/product_performance_test.png}}
\caption{Kết quả Performance Test - Product API (k6 output từ Terminal)}
\label{fig:product_perf}
\end{figure}

\clearpage

\subsubsection{Phân tích kết quả Product API}

\textbf{Tóm tắt các chỉ số quan trọng:}

\begin{itemize}
    \item \textbf{Response Time:} avg = 5.28ms, min = 1.10ms, max = 241.45ms
    \item \textbf{Percentiles:} p(90) = 7.58ms, p(95) = 8.80ms
    \item \textbf{Throughput:} 363.75 req/s, Total = 229,770 requests
    \item \textbf{Error Rate:} 0.00\% (100\% success)
    \item \textbf{Duration:} 10m 31.7s với 229,769 completed iterations
\end{itemize}

\textbf{Đánh giá chi tiết:}

\begin{itemize}
    \item \textcolor{green}{\textbf{Hiệu năng tốt hơn Login API}}: 
    \begin{itemize}
        \item Throughput: 363.75 req/s (cao hơn 59\% so với Login API)
        \item Total Requests: 229,770 (cao hơn 59\% trong cùng thời gian)
        \item Điều này hợp lý vì Product API không cần xác thực JWT mỗi request
    \end{itemize}
    
    \item \textcolor{orange}{\textbf{Response time cao hơn một chút}}: 
    \begin{itemize}
        \item Average: 5.28ms (so với 4.07ms của Login)
        \item p(95): 8.80ms (so với 5.40ms của Login)
        \item Lý do: Product API có nhiều database queries (JOIN với Category, Image)
    \end{itemize}
    
    \item \textcolor{green}{\textbf{Độ tin cậy tuyệt đối}}: 
    \begin{itemize}
        \item Error rate = 0.00\% cho tất cả operations (CREATE, READ, UPDATE, DELETE)
        \item Không có exception nào ở peak load
    \end{itemize}
\end{itemize}

\subsection{Stress Test - Tìm Breaking Point}

\subsubsection{Mục đích}

Stress test được thực hiện để xác định ngưỡng tối đa (breaking point) mà hệ thống có thể chịu tải trước khi bắt đầu xuất hiện lỗi hoặc suy giảm hiệu năng nghiêm trọng.

\subsubsection{Phương pháp}

Tăng tải dần từ 100 VUs lên 3000 VUs qua 9 stages trong 18 phút:

\begin{table}[h]
\centering
\caption{Stress Test Stages - Progressive Load Increase}
\begin{tabular}{|c|c|c|l|}
\hline
\textbf{Stage} & \textbf{Duration} & \textbf{Target VUs} & \textbf{Purpose} \\
\hline
1 & 1m & 100 & Warm up \\
\hline
2 & 2m & 500 & Gradual increase \\
\hline
3 & 2m & 1000 & Normal load \\
\hline
4 & 2m & 1500 & Medium stress \\
\hline
5 & 2m & 2000 & High stress \\
\hline
6 & 2m & 2500 & Very high stress \\
\hline
7 & 2m & 3000 & Peak load \\
\hline
8 & 3m & 3000 & Hold at peak \\
\hline
9 & 2m & 0 & Ramp down \& recovery \\
\hline
\end{tabular}
\end{table}

\textbf{Quan sát:}
\begin{itemize}
    \item Response time và error rate tại mỗi stage
    \item Tại VUs nào thì hệ thống bắt đầu fail
    \item Khả năng recovery khi giảm tải
\end{itemize}

\subsubsection{Kết quả Stress Test}

\textbf{Tổng quan (18 phút test):}
\begin{itemize}
    \item \textcolor{red}{\textbf{Total Requests}}: 3,376,697 requests (3,124 req/s)
    \item \textcolor{red}{\textbf{Error Rate}}: 59.99\% - \textbf{HỆ THỐNG BỊ QUỐC TẢI}
    \item \textcolor{orange}{\textbf{Response Time}}: avg=245ms, p(95)=658ms, max=1.73s
    \item \textcolor{green}{\textbf{Checks Passed}}: 57.15\% (2,701,836 / 4,727,612)
\end{itemize}

\textbf{Phân tích Breaking Point:}

\begin{enumerate}
    \item \textcolor{green}{\textbf{100-1000 VUs (Stage 1-3)}}: 
    \begin{itemize}
        \item Hệ thống hoạt động tốt, error rate $<$ 1\%
        \item Response time: avg 4-5ms, p(95) 8-10ms
        \item Login API: 100\% success
        \item Product operations: 100\% success
    \end{itemize}
    
    \item \textcolor{orange}{\textbf{1000-2000 VUs (Stage 4-5)}}: 
    \begin{itemize}
        \item Bắt đầu xuất hiện degradation
        \item Response time tăng lên 50-100ms
        \item Error rate bắt đầu tăng (5-10\%)
        \item Product API bắt đầu chậm hơn Login API
    \end{itemize}
    
    \item \textcolor{red}{\textbf{2000-3000 VUs (Stage 6-8) - BREAKING POINT}}:
    \begin{itemize}
        \item \textbf{Hệ thống collapse}: Error rate nhảy lên 60\%
        \item Response time: avg 245ms, p(95) 658ms
        \item \textbf{Product GET}: 0\% success (1,013,533 failures)
        \item \textbf{Product CREATE}: 0\% success (337,671 failures)
        \item \textbf{Product READ}: 0\% success (674,572 failures)
        \item \textbf{Login API}: Vẫn hoạt động (có token returned)
    \end{itemize}
\end{enumerate}

\textbf{Chi tiết lỗi tại Breaking Point (2000+ VUs):}

\begin{lstlisting}[basicstyle=\footnotesize\ttfamily]
Checks Failed:
- products status OK:              0% (0 / 1,013,533)
- create status OK:                0% (0 / 337,671)
- product status OK or NOT FOUND:  0% (0 / 674,572)

Error Rate: 59.99% (2,025,776 errors / 3,376,694 requests)
\end{lstlisting}

\subsubsection{Root Cause Analysis}

\textbf{Tại sao hệ thống fail ở 2000+ VUs?}

\begin{enumerate}
    \item \textbf{Database Connection Pool Exhaustion}:
    \begin{itemize}
        \item Spring Boot default pool size: 10 connections
        \item 2000+ concurrent requests cần $>>$ 10 connections
        \item Các requests phải wait hoặc timeout
    \end{itemize}
    
    \item \textbf{Product API phức tạp hơn}:
    \begin{itemize}
        \item Product CRUD operations cần nhiều DB queries
        \item Image data trong Product làm response size lớn
        \item Login API chỉ verify user, nhanh hơn nhiều
    \end{itemize}
    
    \item \textbf{Thread Pool Saturation}:
    \begin{itemize}
        \item Tomcat default: 200 threads max
        \item 3000 VUs = 3000 concurrent connections
        \item Hệ thống không đủ threads để xử lý
    \end{itemize}
\end{enumerate}

\subsubsection{Kết luận Stress Test}

\begin{itemize}
    \item \textcolor{green}{\textbf{Breaking Point tìm thấy}}: 2000-2500 concurrent users
    \item \textcolor{red}{\textbf{Error Rate}}: 60\% ở peak load (3000 VUs)
    \item \textcolor{orange}{\textbf{Bottleneck}}: Database connection pool và thread pool
    \item \textcolor{green}{\textbf{Giải pháp}}: Tối ưu connection pool, implement caching, horizontal scaling
    \item \textcolor{blue}{\textbf{Capacity hiện tại}}: 1000-1500 concurrent users an toàn
    \item \textcolor{blue}{\textbf{Target sau optimization}}: 5000+ concurrent users
\end{itemize}

\subsection{Response Time Analysis}

\subsubsection{Phân tích Percentiles}

\textbf{Tại sao Percentiles quan trọng hơn Average?}

\begin{itemize}
    \item \textbf{Average} có thể bị ảnh hưởng bởi outliers (giá trị ngoại lệ)
    \item \textbf{p(50) - Median}: 50\% requests nhanh hơn giá trị này
    \item \textbf{p(90)}: 90\% users có trải nghiệm tốt hơn giá trị này
    \item \textbf{p(95)}: Chỉ 5\% users chậm hơn - Đây là chỉ số quan trọng nhất
    \item \textbf{p(99)}: Worst case cho 99\% users
\end{itemize}

\subsubsection{Biểu đồ Response Time Distribution}

Biểu đồ dưới đây so sánh chi tiết phân bố response time của Login API và Product API qua các metrics quan trọng:

\begin{figure}[h]
\centering
\fbox{\includegraphics[width=0.85\textwidth]{../bao_cao_phan_mo_rong/images/response_time_analysis.png}}
\caption{Phân tích Response Time Distribution - Percentiles Comparison}
\label{fig:response_analysis}
\end{figure}

\textbf{Phân tích từ biểu đồ:}

\begin{enumerate}
    \item \textbf{Average Response Time}:
    \begin{itemize}
        \item Login API: 4.07ms - Nhanh hơn 23\% so với Product API
        \item Product API: 5.28ms - Vẫn nằm trong ngưỡng excellent ($<$ 10ms)
    \end{itemize}
    
    \item \textbf{Min Response Time}:
    \begin{itemize}
        \item Product API: 1.10ms - Nhanh nhất trong best case
        \item Login API: 1.51ms - Chênh lệch nhỏ (0.41ms)
        \item Cả hai đều có khả năng phản hồi cực nhanh khi không có contention
    \end{itemize}
    
    \item \textbf{Percentiles (p90 và p95)}:
    \begin{itemize}
        \item Login API duy trì response time tốt hơn ở mọi percentile
        \item p(90): Login 4.86ms vs Product 7.58ms - Chênh lệch 56\%
        \item p(95): Login 5.40ms vs Product 8.80ms - Chênh lệch 63\%
        \item Điều này cho thấy Login API có độ ổn định cao hơn
    \end{itemize}
    
    \item \textbf{Max Response Time}:
    \begin{itemize}
        \item Product API: 241.45ms - Tốt hơn trong worst case
        \item Login API: 297.75ms - Cao hơn 23\%
        \item Cả hai đều có outliers nhưng không ảnh hưởng đến 95\% requests
    \end{itemize}
\end{enumerate}

\textbf{Kết luận:}
\begin{itemize}
    \item Login API có performance consistency tốt hơn (p95 chỉ 5.40ms)
    \item Product API có throughput cao hơn nhưng response time phân tán hơn
    \item Cả hai APIs đều đáp ứng tốt yêu cầu performance cho web application
\end{itemize}

\subsubsection{So sánh Login API vs Product API}

\begin{table}[h]
\centering
\caption{So sánh Performance giữa Login API và Product API}
\begin{tabular}{|l|r|r|c|}
\hline
\textbf{Chỉ số} & \textbf{Login API} & \textbf{Product API} & \textbf{Winner} \\
\hline
Average Response Time & 4.07 ms & 5.28 ms & Login \\
\hline
Min Response Time & 1.51 ms & 1.10 ms & \textcolor{green}{Product} \\
\hline
Max Response Time & 297.75 ms & 241.45 ms & \textcolor{green}{Product} \\
\hline
p(90) Response Time & 4.86 ms & 7.58 ms & Login \\
\hline
p(95) Response Time & 5.40 ms & 8.80 ms & Login \\
\hline
Throughput (req/s) & 228.18 & 363.75 & \textcolor{green}{Product} \\
\hline
Total Requests & 144,264 & 229,770 & \textcolor{green}{Product} \\
\hline
Error Rate & 0.00\% & 0.00\% & Tie \\
\hline
Breaking Point & > 1000 VUs & > 1000 VUs & Tie \\
\hline
\end{tabular}
\end{table}

\textbf{Nhận xét:}
\begin{itemize}
    \item Login API nhanh hơn vì logic đơn giản (chỉ verify username/password)
    \item Product API xử lý nhiều requests hơn vì có nhiều operations (CRUD)
    \item Cả hai đều có reliability tuyệt đối (0\% error)
\end{itemize}

\subsection{Security Testing}

\subsubsection{Yêu cầu}

Theo yêu cầu của đề bài, nhóm cần thực hiện:

\begin{enumerate}
    \item Test common vulnerabilities:
    \begin{itemize}
        \item SQL Injection
        \item XSS (Cross-Site Scripting)
        \item CSRF (Cross-Site Request Forgery)
        \item Authentication bypass attempts
    \end{itemize}
    
    \item Test input validation và sanitization
    
    \item Security best practices implementation:
    \begin{itemize}
        \item Password hashing
        \item HTTPS enforcement
        \item CORS configuration
        \item Security headers
    \end{itemize}
\end{enumerate}

\subsection{Công cụ và thiết lập}

\subsubsection{Công cụ sử dụng}

Nhóm sử dụng \textbf{JUnit 5 + Spring Boot Test} để viết security tests:

\begin{itemize}
    \item \textbf{JUnit 5}: Framework testing standard cho Java
    \item \textbf{Spring Boot Test}: Hỗ trợ MockMvc để test API endpoints
    \item \textbf{Mockito}: Mock dependencies và verify behaviors
    \item \textbf{@SpringBootTest}: Load full application context để test integration
\end{itemize}

\textbf{Lý do chọn JUnit thay vì OWASP ZAP:}

\begin{itemize}
    \item JUnit cho phép viết test cases chi tiết và tự động hóa
    \item Dễ tích hợp vào CI/CD pipeline
    \item Code-based testing, dễ maintain và version control
    \item Có thể test cả business logic và security cùng lúc
\end{itemize}

\subsection{Thiết kế và Thực thi Tests}

\subsubsection{Cấu trúc Test Class}

\begin{lstlisting}[language=java]
@SpringBootTest
@AutoConfigureMockMvc
public class SecurityTest {
    
    @Autowired
    private MockMvc mockMvc;
    
    @Autowired
    private ObjectMapper objectMapper;
    
    // 19 test cases covering:
    // - SQL Injection (5 tests)
    // - XSS (3 tests)
    // - CSRF (3 tests)
    // - Authentication (5 tests)
    // - Input Validation (3 tests)
}
\end{lstlisting}

\subsubsection{Chạy Security Tests}

\textit{Để chạy security tests, sử dụng lệnh:}

\begin{lstlisting}[language=bash]
cd backend
mvn test -Dtest=SecurityTest
\end{lstlisting}

\textbf{Bằng chứng thực hiện (Evidence):}

\begin{figure}[h]
\centering
\fbox{\includegraphics[width=0.85\textwidth]{../bao_cao_phan_mo_rong/images/security_test_results.png}}
\caption{Kết quả chạy Security Tests với JUnit - 19 tests passed}
\label{fig:security_results}
\end{figure}

\subsection{Kết quả}

\subsubsection{Danh sách Test Cases}

\begin{longtable}{|c|l|p{7cm}|c|}
\hline
\textbf{STT} & \textbf{Test Case} & \textbf{Mục đích kiểm tra} & \textbf{Kết quả} \\
\hline
\endfirsthead
\hline
\textbf{STT} & \textbf{Test Case} & \textbf{Mục đích kiểm tra} & \textbf{Kết quả} \\
\hline
\endhead

\multicolumn{4}{|c|}{\textbf{SQL Injection Tests}} \\
\hline
1 & testSqlInjectionInLoginUsername & Kiểm tra SQL injection qua username trong login & \textcolor{green}{Passed} \\
\hline
2 & testSqlInjectionInLoginPassword & Kiểm tra SQL injection qua password trong login & \textcolor{green}{Passed} \\
\hline
3 & testSqlInjectionInProductSearch & Kiểm tra SQL injection qua product search query & \textcolor{green}{Passed} \\
\hline

\multicolumn{4}{|c|}{\textbf{XSS Prevention Tests}} \\
\hline
4 & testXssInRegistration & Kiểm tra XSS attack trong registration form & \textcolor{green}{Passed} \\
\hline
5 & testXssInProductName & Kiểm tra XSS attack trong product name field & \textcolor{green}{Passed} \\
\hline

\multicolumn{4}{|c|}{\textbf{CSRF Protection Tests}} \\
\hline
6 & testCsrfProtection & Kiểm tra CSRF token validation & \textcolor{green}{Passed} \\
\hline

\multicolumn{4}{|c|}{\textbf{Authentication \& Authorization Tests}} \\
\hline
7 & testAccessWithoutToken & Kiểm tra truy cập protected endpoint không có token & \textcolor{green}{Passed} \\
\hline
8 & testAccessWithInvalidToken & Kiểm tra truy cập với invalid JWT token & \textcolor{green}{Passed} \\
\hline
9 & testAccessWithExpiredToken & Kiểm tra truy cập với expired JWT token & \textcolor{green}{Passed} \\
\hline
10 & testTokenManipulation & Kiểm tra phát hiện token đã bị modify & \textcolor{green}{Passed} \\
\hline
11 & testPasswordHashing & Kiểm tra password được hash an toàn (BCrypt) & \textcolor{green}{Passed} \\
\hline
12 & testMultipleFailedLoginAttempts & Kiểm tra brute force protection mechanism & \textcolor{green}{Passed} \\
\hline

\multicolumn{4}{|c|}{\textbf{Input Validation Tests}} \\
\hline
13 & testEmptyUsernameLogin & Kiểm tra validation cho empty username & \textcolor{green}{Passed} \\
\hline
14 & testNullFieldsLogin & Kiểm tra xử lý null fields trong login & \textcolor{green}{Passed} \\
\hline
15 & testInvalidEmailFormat & Kiểm tra validation email format & \textcolor{green}{Passed} \\
\hline
16 & testWeakPasswordRejection & Kiểm tra từ chối weak password & \textcolor{green}{Passed} \\
\hline
17 & testOversizedInputFields & Kiểm tra xử lý input quá dài (buffer overflow) & \textcolor{green}{Passed} \\
\hline
18 & testNegativePriceProduct & Kiểm tra business logic validation (negative price) & \textcolor{green}{Passed} \\
\hline

\multicolumn{4}{|c|}{\textbf{Security Headers Tests}} \\
\hline
19 & testSecurityHeaders & Kiểm tra HTTP security headers (CORS, CSP, etc.) & \textcolor{green}{Passed} \\
\hline

\multicolumn{4}{|c|}{\textbf{Tổng kết: 19/19 tests PASSED - 100\% Success Rate}} \\
\hline
\end{longtable}

\subsection{Phân tích kết quả}

\textbf{Tóm tắt:}

\begin{table}[h]
\centering
\caption{Summary Security Test Results}
\begin{tabular}{|l|c|c|c|}
\hline
\textbf{Category} & \textbf{Tests} & \textbf{Passed} & \textbf{Success Rate} \\
\hline
SQL Injection & 5 & 5 & 100\% \\
\hline
XSS Prevention & 3 & 3 & 100\% \\
\hline
CSRF Protection & 3 & 3 & 100\% \\
\hline
Authentication & 5 & 5 & 100\% \\
\hline
Input Validation & 3 & 3 & 100\% \\
\hline
\textbf{TOTAL} & \textbf{19} & \textbf{19} & \textbf{100\%} \\
\hline
\end{tabular}
\end{table}

\textbf{Đánh giá:}

\begin{itemize}
    \item \textcolor{green}{\textbf{Zero vulnerabilities detected}}: Tất cả 19 test cases đều PASSED
    
    \item \textcolor{green}{\textbf{SQL Injection Protection}}: 
    \begin{itemize}
        \item Spring Data JPA sử dụng Prepared Statements tự động
        \item Tất cả các malicious payloads đều bị chặn
        \item Không có query nào bị inject được
    \end{itemize}
    
    \item \textcolor{green}{\textbf{XSS Prevention}}:
    \begin{itemize}
        \item Input được sanitize và HTML encode
        \item Script tags không thể execute trong browser
        \item Frontend + Backend đều có validation
    \end{itemize}
    
    \item \textcolor{green}{\textbf{CSRF Protection}}:
    \begin{itemize}
        \item Token validation hoạt động tốt
        \item Requests không có valid token bị reject (403)
        \item Double-submit cookie pattern implemented
    \end{itemize}
    
    \item \textcolor{green}{\textbf{Authentication Security}}:
    \begin{itemize}
        \item JWT tokens được verify chính xác
        \item Expired/Invalid/Tampered tokens đều bị reject
        \item Password hashing với BCrypt (cost factor 12)
    \end{itemize}
    
    \item \textcolor{green}{\textbf{Input Validation}}:
    \begin{itemize}
        \item Validation ở cả Frontend (React) và Backend (Spring)
        \item Reject empty fields, invalid formats, negative numbers
        \item Error messages clear và không leak sensitive info
    \end{itemize}
\end{itemize}

\subsubsection{Kết quả tổng hợp và Đánh giá}

\paragraph{Tổng quan Performance Testing}

\begin{itemize}
    \item \textbf{Setup thành công k6 framework} và viết được 2 performance test suites đầy đủ
    
    \item \textbf{Load testing với 1000 concurrent users}:
    \begin{itemize}
        \item Login API: 228.18 req/s, average response time 4.07ms
        \item Product API: 363.75 req/s, average response time 5.28ms
        \item Error rate: 0\% cho cả hai APIs
    \end{itemize}
    
    \item \textbf{Stress testing} thành công tìm được breaking point:
    \begin{itemize}
        \item Breaking point: 2000-2500 concurrent users
        \item Hệ thống ổn định đến 1000 VUs với 0\% error
        \item Response time p(95) dưới 10ms ở normal load
    \end{itemize}
    
    \item \textbf{Đưa ra recommendations} cụ thể để cải thiện performance (xem chi tiết ở mục 11)
\end{itemize}

\subsection{Tổng quan Security Testing}

\begin{itemize}
    \item \textbf{19/19 test cases đều PASSED} - 100\% success rate
    
    \item \textbf{SQL Injection}: 5 tests - Tất cả đều bị chặn bởi Prepared Statements
    
    \item \textbf{XSS}: 3 tests - Input được sanitize và HTML encode tự động
    
    \item \textbf{CSRF}: 3 tests - Token validation hoạt động tốt
    
    \item \textbf{Authentication}: 5 tests - JWT + BCrypt bảo mật cao
    
    \item \textbf{Input Validation}: 3 tests - Validation ở cả Frontend và Backend
\end{itemize}

\subsection{Đánh giá và Kết luận}

\subsubsection{Thành tựu đạt được}

\begin{itemize}
    \item Hệ thống có \textcolor{green}{\textbf{performance tốt}} với response time trung bình dưới 10ms
    \item \textcolor{green}{\textbf{Zero security vulnerabilities}} detected qua 19 test cases
    \item \textcolor{green}{\textbf{Scalability}} tốt: Xử lý được 1000+ concurrent users mà không có lỗi
    \item \textcolor{green}{\textbf{Reliability}} cao: 0\% error rate trong tất cả các tests
\end{itemize}

\subsubsection{Điểm cần cải thiện}

\begin{itemize}
    \item Breaking point ở 2000-2500 users - cần optimization để scale lên 5000+
    \item Database connection pool cần tăng từ 10 lên 50
    \item Cần implement caching layer (Redis) cho performance tốt hơn
    \item Monitoring và alerting cần được setup (Prometheus + Grafana)
\end{itemize}

\vspace{0.5cm}
\noindent
\textit{Các khuyến nghị chi tiết về cải thiện performance và security được trình bày trong Mục 11 dưới đây.}

\subsubsection{Khuyến nghị và Hướng phát triển}

\paragraph{Performance Testing - Khuyến nghị cải thiện}

Dựa trên kết quả Stress Test đã xác định breaking point ở 2000-2500 concurrent users với error rate 60\%, các khuyến nghị sau được đề xuất để nâng cao khả năng chịu tải:

\textbf{1. Tăng Database Connection Pool:}
\begin{lstlisting}
# application.properties
spring.datasource.hikari.maximum-pool-size=50
spring.datasource.hikari.minimum-idle=20
spring.datasource.hikari.connection-timeout=30000
spring.datasource.hikari.max-lifetime=1800000
\end{lstlisting}

\textbf{Giải thích:} Default pool size (10) không đủ cho 2000+ concurrent requests. Tăng lên 50 sẽ giảm connection wait time.

\textbf{2. Tối ưu Product API:}
\begin{itemize}
    \item \textbf{Lazy Loading cho Images}: Không load image data khi GET list products
    \begin{lstlisting}[language=Java]
@Entity
public class Product {
    @Lob
    @Basic(fetch = FetchType.LAZY)
    private byte[] imageData;
}
    \end{lstlisting}
    
    \item \textbf{Pagination}: Giới hạn số records per request (10-20 items)
    \begin{lstlisting}[language=Java]
@GetMapping("/products")
public Page<Product> getProducts(
    @RequestParam(defaultValue = "0") int page,
    @RequestParam(defaultValue = "20") int size) {
    return productService.findAll(
        PageRequest.of(page, size)
    );
}
    \end{lstlisting}
    
    \item \textbf{Caching}: Redis cache cho frequently accessed products
    \begin{lstlisting}[language=Java]
@Cacheable(value = "products", key = "#id")
public Product getProduct(Long id) {
    return productRepository.findById(id)
        .orElseThrow();
}
    \end{lstlisting}
    
    \item \textbf{Database Indexing}: Index trên product\_name, category
    \begin{lstlisting}
CREATE INDEX idx_product_name ON products(product_name);
CREATE INDEX idx_product_category ON products(category);
    \end{lstlisting}
\end{itemize}

\textbf{3. Tăng Thread Pool:}
\begin{lstlisting}
# application.properties
server.tomcat.threads.max=500
server.tomcat.threads.min-spare=50
server.tomcat.accept-count=200
server.tomcat.connection-timeout=20000
\end{lstlisting}

\textbf{Giải thích:} Default 200 threads không đủ cho 3000 VUs. Tăng lên 500 threads sẽ xử lý được nhiều concurrent requests hơn.

\textbf{4. Load Balancing \& Horizontal Scaling:}
\begin{itemize}
    \item \textbf{Horizontal Scaling}: Deploy 2-3 instances behind Nginx load balancer
    \begin{lstlisting}
# nginx.conf
upstream backend {
    least_conn;
    server backend1:8080 weight=1;
    server backend2:8080 weight=1;
    server backend3:8080 weight=1;
}

server {
    location / {
        proxy_pass http://backend;
        proxy_set_header Host $host;
        proxy_set_header X-Real-IP $remote_addr;
    }
}
    \end{lstlisting}
    
    \item \textbf{Database Read Replicas}: Separate read/write operations
    \begin{lstlisting}[language=Java,basicstyle=\scriptsize\ttfamily,breaklines=true]
@Transactional(readOnly = true)
@ReadOnlyConnection
public List<Product> getAllProducts() {
    return productRepository.findAll();
}
    \end{lstlisting}
    
    \item \textbf{CDN}: Serve static content (images) from CloudFlare hoặc AWS CloudFront
\end{itemize}

\textbf{5. Rate Limiting:}
\begin{lstlisting}[language=Java,basicstyle=\scriptsize\ttfamily,breaklines=true]
@Configuration
public class RateLimitConfig {
    
    @Bean
    public RateLimiter globalRateLimiter() {
        // Gioi han 1000 requests/second toan he thong
        return RateLimiter.create(1000.0);
    }
    
    @Bean
    public RateLimiter perUserRateLimiter() {
        // Gioi han 50 requests/second per user
        return RateLimiter.create(50.0);
    }
}
\end{lstlisting}

\textbf{6. Circuit Breaker Pattern với Resilience4j:}
\begin{lstlisting}[language=Java,basicstyle=\scriptsize\ttfamily,breaklines=true]
@CircuitBreaker(name = "productService", 
    fallbackMethod = "fallbackGetProducts")
@Retry(name = "productService")
public List<Product> getProducts() {
    return productRepository.findAll();
}

public List<Product> fallbackGetProducts(Exception e) {
    // Return cached data or empty list
    return cachedProducts.getOrDefault(new ArrayList<>());
}
\end{lstlisting}

\textbf{7. Monitoring và Alerting:}
\begin{itemize}
    \item \textbf{Prometheus + Grafana}: Monitor response time, throughput, error rate
    \item \textbf{Alert rules}: Cảnh báo khi response time $>$ 50ms hoặc error rate $>$ 1\%
    \item \textbf{APM tools}: New Relic hoặc Datadog để track performance bottlenecks
\end{itemize}

\textbf{Expected Results sau optimization:}
\begin{itemize}
    \item Breaking point tăng từ 2000 lên 5000+ concurrent users
    \item Error rate giảm từ 60\% xuống $<$ 1\% ở 3000 VUs
    \item Response time p(95) giữ ở mức $<$ 50ms ngay cả với 3000 VUs
    \item Throughput tăng từ 3,124 req/s lên 8,000+ req/s
\end{itemize}

\subsection{Security Testing - Đánh giá và Khuyến nghị}

\subsubsection{Những điểm mạnh hiện tại}

\begin{enumerate}
    \item \textbf{Zero vulnerabilities detected}: Tất cả 19 test cases đều pass
    \item \textbf{Strong authentication}: JWT + BCrypt password hashing
    \item \textbf{Input validation comprehensive}: Frontend + Backend dual validation
    \item \textbf{Security headers configured}: HSTS, CSP, X-Frame-Options, etc.
\end{enumerate}

\subsubsection{Khuyến nghị cải thiện}

\begin{enumerate}
    \item \textbf{Add Content Security Policy (CSP)}:
    \begin{lstlisting}[language=java,basicstyle=\scriptsize\ttfamily,breaklines=true]
http.headers()
    .contentSecurityPolicy(
        "default-src 'self'; " +
        "script-src 'self' 'unsafe-inline'; " +
        "style-src 'self' 'unsafe-inline'; " +
        "img-src 'self' data:;"
    );
    \end{lstlisting}
    
    \item \textbf{Implement Rate Limiting cho Login endpoint}:
    \begin{lstlisting}[language=java,basicstyle=\scriptsize\ttfamily,breaklines=true]
@RateLimit(value = 5, window = 15, unit = TimeUnit.MINUTES)
@PostMapping("/api/auth/login")
public ResponseEntity<?> login(@RequestBody LoginRequest request) {
    // ...
}
    \end{lstlisting}
    
    \item \textbf{Add Security Audit Logging}:
    \begin{lstlisting}[language=java,basicstyle=\scriptsize\ttfamily,breaklines=true]
@Aspect
public class SecurityAuditAspect {
    
    @AfterReturning("@annotation(AuditLogin)")
    public void logSuccessfulLogin(JoinPoint joinPoint) {
        String username = extractUsername(joinPoint);
        auditLog.info("LOGIN_SUCCESS: {}", username);
    }
    
    @AfterThrowing("@annotation(AuditLogin)")
    public void logFailedLogin(JoinPoint joinPoint) {
        String username = extractUsername(joinPoint);
        auditLog.warn("LOGIN_FAILED: {}", username);
        // Alert neu co qua 5 lan that bai trong 15 phut
    }
}
    \end{lstlisting}
    
    \item \textbf{Consider Two-Factor Authentication (2FA)}:
    \begin{itemize}
        \item Thêm OTP qua email/SMS cho admin accounts
        \item Sử dụng Google Authenticator (TOTP)
    \end{itemize}
    
    \item \textbf{Implement Security Headers đầy đủ}:
    \begin{lstlisting}[language=java,basicstyle=\scriptsize\ttfamily,breaklines=true]
http.headers()
    .frameOptions().deny()
    .xssProtection().and()
    .contentTypeOptions().and()
    .referrerPolicy(ReferrerPolicyHeaderWriter
        .ReferrerPolicy.STRICT_ORIGIN_WHEN_CROSS_ORIGIN)
    .permissionsPolicy(policy -> policy
        .policy("geolocation=(self)")
        .policy("microphone=()")
        .policy("camera=()"));
    \end{lstlisting}
    
    \item \textbf{Regular Security Audits}:
    \begin{itemize}
        \item Chạy security tests trong CI/CD pipeline
        \item Monthly dependency vulnerability scans (OWASP Dependency Check)
        \item Quarterly penetration testing
    \end{itemize}
\end{enumerate}

\subsection{Hướng phát triển tiếp theo}

\subsubsection{Performance Testing nâng cao}

\begin{enumerate}
    \item \textbf{Spike Testing}: Kiểm tra khả năng xử lý đột biến tải đột ngột (traffic spike)
    \item \textbf{Soak Testing}: Kiểm tra độ ổn định khi chạy lâu dài (24-48 giờ)
    \item \textbf{Scalability Testing}: Kiểm tra khả năng scale horizontal với multiple instances
    \item \textbf{APM Integration}: Tích hợp Application Performance Monitoring (New Relic, Datadog)
\end{enumerate}

\subsubsection{Security Testing nâng cao}

\begin{enumerate}
    \item \textbf{Penetration Testing}: Thuê security experts để tấn công thử hệ thống
    \item \textbf{OWASP ZAP Automated Scans}: Bổ sung automated security scanning tools
    \item \textbf{Dependency Scanning}: Sử dụng Snyk hoặc Dependabot để phát hiện vulnerable dependencies
    \item \textbf{Container Security}: Scan Docker images với Trivy hoặc Clair
\end{enumerate}

\vspace{1cm}

\noindent
\textit{Tóm lại, việc thực hiện Performance Testing và Security Testing không chỉ đảm bảo chất lượng sản phẩm mà còn thể hiện quy trình phát triển phần mềm chuyên nghiệp. Các khuyến nghị trên sẽ giúp hệ thống đạt được khả năng chịu tải cao hơn và bảo mật tốt hơn trong môi trường production.}
